\begin{flushleft}
  % Ubah kalimat berikut sesuai dengan nama departemen dan fakultas
  \textbf{Departemen Teknik Komputer - FTEIC}\\
  \textbf{Institut Teknologi Sepuluh Nopember}\\
\end{flushleft}

\begin{center}
  % Ubah detail mata kuliah berikut sesuai dengan yang ditentukan oleh departemen
  \underline{\textbf{EC184701  - PRA TUGAS AKHIR - 2 SKS}}
\end{center}

\begin{adjustwidth}{-0.2cm}{}
  \begin{tabular}{lcp{0.7\linewidth}}

    % Ubah kalimat-kalimat berikut sesuai dengan nama dan NRP mahasiswa
    Nama Mahasiswa &:& Farhan Fadhurrahman \\
    NRP &:& 0721 18 4000 0066 \\

    % Ubah kalimat berikut sesuai dengan semester pengajuan proposal
    Semester &:& Ganjil 2021/2022 \\

    % Ubah kalimat-kalimat berikut sesuai dengan nama-nama dosen pembimbing
    Dosen Pembimbing &:& 1.  Dr. Supeno Mardi Susiki Nugroho, S.T., M.T. \\
    & & 2. Reza Fuad Rachmadi, S.T., M.T., Ph.D \\

    % Ubah kalimat berikut sesuai dengan judul tugas akhir
    Judul Tugas Akhir &:& \textbf{Evaluasi Performa Sistem Deteksi Penggunaan Helm pada Pengendara Motor Menggunakan YOLO}\\

    Uraian Tugas Akhir &:& \\
  \end{tabular}
\end{adjustwidth}

% Ubah paragraf berikut sesuai dengan uraian dari tugas akhir
Helm merupakan sebuah alat pelindung yang berfungsi untuk melindungi kepala saat
terjadi benturan. Pada sebuah penelitian didapatkan bahwa menggunakan Helm saat
mengendarai sepeda motor, dapat mengurangi resiko cedera dari kecelakaan lalu lintas
hingga 42\% \citep{helmetUse}. Berdasarkan data dari Badan Pusat Statistik (BPS), jumlah korban
meninggal dalam kasus kecelakaan lalu lintas di Indonesia sebanyak 25.671 jiwa pada
tahun 2019. Pertumbuhan jumlah korban jiwa setiap tahun terus bertambah dengan
persentase hingga 1.41\% \citep{transportasi2019}. Dari latar belakang tersebut, masalah yang dapat diambil
adalah deteksi pengendara Sepeda Motor yang tidak menggunakan helm saat berkendara
yang tidak terdeteksi secara menyeluruh. Oleh sebab itu, diperlukan sebuah sistem untuk
mendeteksi pengendara Sepeda Motor yang tidak menggunakan helm untuk dikembangkan
dalam mendeteksi pelanggar yang tidak menggunakan helm. Dari latar belakang tersebut,
masalah yang dapat diambil adalah deteksi pengendara Sepeda Motor yang tidak
menggunakan helm saat berkendara yang tidak terdeteksi secara menyeluruh. Oleh sebab
itu, diperlukan sebuah sistem untuk mendeteksi pengendara Sepeda Motor yang tidak
menggunakan helm untuk dikembangkan dalam mendeteksi pelanggar yang tidak
menggunakan helm. Tujuan dari penelitian ini adalah mengembangkan dan mengevaluasi performa dari sistem pendeteksi
penggunaan helm pada pengendara Sepeda Motor dengan metode \emph{You Only Look Once} (YOLO).
\vspace{1ex}

\begin{flushright}
  % Ubah kalimat berikut sesuai dengan tempat, bulan, dan tahun penulisan
  Surabaya, Desember 2021
\end{flushright}
\vspace{1ex}

\begin{center}

  \begin{multicols}{2}

    Dosen Pembimbing 1
    \vspace{12ex}

    % Ubah kalimat-kalimat berikut sesuai dengan nama dan NIP dosen pembimbing pertama
    \underline{Dr. Supeno Mardi Susiki Nugroho, S.T., M.T.} \\
    NIP. 197003131995121001

    \columnbreak

    Dosen Pembimbing 2
    \vspace{12ex}

    % Ubah kalimat-kalimat berikut sesuai dengan nama dan NIP dosen pembimbing kedua
    \underline{Reza Fuad Rachmadi, S.T., M.T., Ph.D} \\
    NIP. 198504032012121000

  \end{multicols}
  \vspace{6ex}

  Mengetahui, \\
  % Ubah kalimat berikut sesuai dengan jabatan kepala departemen
  Kepala Departemen Teknik Komputer FTEIC - ITS
  \vspace{12ex}

  % Ubah kalimat-kalimat berikut sesuai dengan nama dan NIP kepala departemen
  \underline{Dr. Supeno Mardi Susiki Nugroho, S.T., M.T. } \\
  NIP. 197003131995121001

\end{center}
