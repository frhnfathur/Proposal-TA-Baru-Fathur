\section{PENDAHULUAN}

\subsection{Latar Belakang}

% Ubah paragraf-paragraf berikut sesuai dengan latar belakang dari tugas akhir
Helm merupakan sebuah alat pelindung yang berfungsi untuk melindungi kepala saat
terjadi benturan. Pada sebuah penelitian didapatkan bahwa menggunakan Helm saat
mengendarai sepeda motor, dapat mengurangi resiko cedera dari kecelakaan lalu lintas
hingga 42\% \citep{helmetUse}. Berdasarkan data dari Badan Pusat Statistik (BPS), jumlah korban
meninggal dalam kasus kecelakaan lalu lintas di Indonesia sebanyak 25.671 jiwa pada
tahun 2019. Pertumbuhan jumlah korban jiwa setiap tahun terus bertambah dengan
persentase hingga 1.41\% \citep{transportasi2019}. Tingginya jumlah korban meninggal dunia pada kecelakaan
lalu lintas diikuti dengan tingginya jenis pelanggaran yang sering dilakukan salah satunya
adalah tidak menggunakan helm oleh pengendara sepeda motor pada saat berkendara \citep{kecelakaan}.

Salah satu penyebab dari tingginya angka kecelakaan pada pengendara sepeda motor adalah keterbatasan
jumlah anggota polisi dalam memantau lalu-lintas sehingga sering terjadi kelalaian pemantauan dan penindakan
pengendara sepeda motor yang tidak menggunakan helm. Oleh karena itu sistem deteksi penggunaan helm pada pengendara
sepeda motor penting, dengan menggunakan teknologi untuk mendapatkan informasi pengendara sepeda motor yang tidak 
menggunakan helm saat berkendara.

Teknologi Visi Komputer adalah sebuah bidang kecerdasan buatan atau Artificial
Intelligence (AI) yang melatih sebuah komputer agar dapat memahami sesuatu dari input
visual. Teknologi Visi Komputer memanfaatkan input dari gambar digital menggunakan
kamera dan jaringan neural atau Neural Network  untuk mengklasifikasikan objek
kemudian dapat mendeteksi sesuatu yang ditangkap oleh komputer. Teknologi Visi
Komputer saat ini sudah banyak diimplementasikan dalam berbagai bidang seperti
administrasi, Kesehatan sampai dengan Transportasi.

\subsection{Rumusan Masalah}

% Ubah paragraf berikut sesuai dengan rumusan masalah dari tugas akhir
Permasalahan yang dapat dirumuskan pada penelitian ini adalah yang pertama deteksi pengendara 
Sepeda Motor yang tidak menggunakan helm saat berkendara tidak terdeteksi secara
menyeluruh. Sehingga diperlukan sebuah sistem dengan performa yang baik untuk mendeteksi pengendara
Sepeda Motor yang tidak menggunakan helm untuk dikembangkan dalam mendeteksi
pelanggar yang tidak menggunakan helm.

\subsection{Penelitian Terkait}

% Ubah paragraf berikut sesuai dengan penelitian lain yang terkait dengan tugas akhir
Hanhe Lin et al dalam artikel penelitiannya yang berjudul \emph{"Helmet Use Detection of Tracked Motorcycles
Using CNN-Based Multi-Task Learning"} pada tahun 2019 mengembangkan sebuah sistem pendeteksi helm
pada pengendara sepeda motor yang berlokasi di beberapa kota negara Myanmar menggunakan metode RetinaNet yang merupakan
sebuah metode \emph{single-stage} untuk mendeteksi sebuah objek. Dalam penilitian ini didapatkan
\emph{mean Average Precision} sebesar 95.3\% \citep{hanhelin}.

Yusuf Umar Hanafi dalam buku tugas akhirnya yang berjudul "Deteksi Penggunaan Helm pada Pengendara Bermotor Berbasis
Deep Learning" pada tahun 2020 membuat sebuah sistem pendeteksi helm menggunakan metode \emph{You Look Only Once} (YOLO). Data input
yang digunakan adalah rekaman video CCTV lalu lintas milik Dinas Perhubungan kota Surabaya. Pengujian dilakukan berdasarkan lokasi,
kondisi, dan objek pelanggar. Pada pengujian ini didapatkan \emph{mean Average Precision} yang cukup baik untuk YOLOv3 \citep{masYusuf}.


\subsection{Tujuan Penelitian}

% Ubah paragraf berikut sesuai dengan tujuan penelitian dari tugas akhir
Tujuan dari penelitian ini adalah:

\begin{enumerate}[nolistsep]

  \item Mengembangkan sistem deteksi penggunaan helm pada pengendara motor menggunakan metode YOLO.
  \item Mengevaluasi performa dari metode YOLO dalam mendeteksi penggunaan helm pada pengendara sepeda motor.

\end{enumerate}
